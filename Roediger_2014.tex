% !TeX root = RJwrapper.tex
\title{R as Platform for the Analysis of dPCR and qPCR Experiments}
\author{by Stefan R\"{o}diger, Micha\l{} Burdukiewicz, Konstantin 
Blagodatskikh and Peter Schierack}

\maketitle

\abstract{
There is an ever-increasing number of publications, which use quantitative PCR 
(qPCR) or digital PCR (dPCR) to elicit fundamentals of biological processes. 
Novel amplification strategies based on quantitative isothermal amplification 
(qIA) start to become more prominent in life sciences and diagnostics. Several 
software solutions have been developed, which are either distributed as closed 
source software or as monolithic block with little freedom to perform highly 
customized analysis procedures. Others and we argue that R is an excellent 
environment for a reproducible and transparent analysis of data. However, for 
newcomers it is often very challenging to master R from reading the manuals and 
FAQs the not so obvious steps. Here we describe exemplary work-flows for the 
analysis of dPCR, qIA or qPCR experiments including the analysis of melting curve 
data. Our analysis relies entirely on R packages available from public 
repositories.
}

\section{Introduction}

The qPCR is the method of choice when a precise quantification of minute DNA 
traces of pathogens or the analysis of gene expression is required 
\citep{peirson_2003}. Numerous technologies have been developed in the past 
years \citep{rodiger_highly_2013, devonshire_2013, viturro_2014, 
rodiger_nucleic_2014, khodakov_2014}. Only few bioanalytical applications had such a 
significant impact on the progress of life sciences and medical sciences as the 
quantitative Polymerase Chain Reaction (qPCR). The scientific community work 
hard in the past two decades to uncover pitfalls of qPCR experiments. This lead 
finally to the development of peer-reviewed analysis algorithms 
\citep{ruijter_2013}, throughout analysed qPCR chemistries \citep{ruijter_2014} 
and guidelines for a proper conduct of qPCR experiments as implemented in the 
MIQE guidelines (minimum information for publication of quantitative real-time 
PCR experiments) \citep{bustin_miqe_2009, huggett_2013}. We share the philosophy 
of the MIQE guidelines to increase experimental transparency for better 
experimental practice and reliable interpretation of qPCR results. In the past 
decades emerged several isothermal amplification technologies, such as helicase 
dependent amplification (HDA). Isothermal amplification was readily combined 
with real-time monitoring technologies (qIA) and is used in various fields like 
diagnostics and point of care testing \citep{rodiger_nucleic_2014}.

R is one of the most used tools in bioinformatics and is known as an early 
adopter of emerging technologies \citep{pabinger_2014}. Recently we published 
the \CRANpkg{dpcR} package at CRAN, which is the first open source software 
package based on R for the analysis of digital PCR (dPCR) experiments. The dPCR 
Technology breaks fundamentally with the previous concept of nucleic acid 
quantification and can be seen as a next generation nucleic acid quantification. 
The key difference between dPCR and traditional PCR lies in the method of 
measuring (absolute) nucleic acids amounts, which yields discrete information
instead of the countinous signal. This is possible after ``clonal DNA 
amplification'' in thousands of small separated partitions (e.g., droplets, nano 
chambers) \citep{huggett_2013, milbury_2014, morley_2014}. Partitions with no 
nucleic acid remain negative and the others turn positive. Selected technologies 
(e.g., OpenArray\textregistered Real-Time PCR System) monitor amplification 
reactions in the chambers in real-time. Cq values are calculated from the 
amplification curves and converted into discrete events by means of positive and 
negative partitions and the absolute quantification of nucleic acids is done by 
Poisson statistics (see \CRANpkg{dpcR} for details).

Most of the commercial and experimental hardware platforms provide means to 
analyse the amplification curve data. Yet, in case of closed source software the 
analysis happens in most cases in a blackbox fashion tied to a specific 
platforms. Often such systems have limitation in the data processing and force 
the used to suboptimal analysis algorithm as discussed by \citet{ruijter_2013}. 
The visualization options are usually limited by the software and not in 
acceptable publication quality. The data processing in spreadsheets is time 
consuming and prone to errors because the advanced mathematical expertise is 
often not available in a laboratory.

The complexity of hardware, wetware and software require expertise to master a 
technical workflow comprising standards for experimental design, generation and 
analysis of data, interpretation of results and reporting 
\citep{huggett_BDQ_2014}. We argue that blackboxs are not necessarily a bad 
thing, but should be avoided wherever possible. A recent study by 
\citet{Duran_2014} exemplified this. Scientific misconduct and fraud have shaken 
the scientific community on several occasions \citep{fang_2012}. In particular 
qPCR is a sensible topic. Therefore, several reasons strongly support to use R 
in science. R provides essential packages to build a highly customized 
workflows, covering: data read-in, data preprocessing, analysis, 
post-processing, visualization and storage. As recently briefly reviewed in 
\citet{pabinger_2014}, numerous R packages have been developed for the analysis 
of qPCR experiments, including: \CRANpkg{kulife}, \CRANpkg{MCMC.qpcr}, 
\CRANpkg{qPCR.CT}, \CRANpkg{DivMelt}, \CRANpkg{qpcR}, \CRANpkg{dpcR}, 
\CRANpkg{chipPCR}, \CRANpkg{MBmca}, \CRANpkg{RDML}, \BIOpkg{nondetects}, 
\BIOpkg{qpcrNorm}, \BIOpkg{HTqPCR}, \BIOpkg{SLqPCR}, \BIOpkg{ddCt}, 
\BIOpkg{EasyqpcR}, \BIOpkg{unifiedWMWqPCR}, \BIOpkg{ReadqPCR}, 
\BIOpkg{NormqPCR}. All the packages are either available from CRAN or 
Bioconductor \citep{gentleman_2004}. The packages can be freely combined in a 
plugin-like architecture. R is instrument independent, cross-platform and 
provides a wide spectrum of calculation options. In particular, visualization of 
experiments is one of R pinnacles. Though the intrinsic properties of R such as 
the naming convention \citep{Baaaath_2012} and use of R's class systems (e.g., 
\strong{S3}, \strong{S4}, reference classes and \strong{R6}) vary considerable 
depending on the package developer preferences there is the common ground to 
track numerical errors in R due to the open source approach. In addition, offers 
the R environment several data sets. R offers various methods for a standardized 
data import/export and exchange. Workflows can be embedded in structures for 
models (e.g., Predictive Model Markup Language (PMML) as proposed by 
\citet{Guazzelli_2009}, open data exchange formats (e.g., XML-based Real-Time 
PCR Data Markup Language (RDML) \citep{lefever_2009}, binary formats 
\citep{michna_2013} or tools provided by the R workspace \citep{RDCT2010c}. 
Therefore, others and we argue that R is suitable for reproducible research 
\citep{Gesmann_2011, Murrell_2012, gandrud_2013, hofmann_2013, Leeper_2014, 
liu_2014}.

The aim of this paper is to show two simple examples. In 
particular, we show how to:
\begin{itemize}
 \item read-in data from a standardized file format,
 \item pre-process the amplification curve data,
 \item calculate specific parameters from the amplification curve data,
 \item calculate the melting temperature,
 \item and report the data.
\end{itemize}

Our workflow effectively follows the principle illustrated in 
Figure~\ref{figure:workflow}. The intent is to aggregate functionalities 
dispersed between various packages and offer a fast insight for novices 
in the analysis of qPCR experiments with R.

\begin{figure}[htbp]
  \centering
%   \includegraphics[clip=true,trim=0.1cm 0.6cm 1cm 1.8cm, width=16cm]{figures/plotCurves.pdf}
  \includegraphics{figures/workflow.png}
  \caption{Exemplary workflow for qPCR experiments in R. Core functionality is 
provided by the R software environment for statistical computing and graphics. 
In our scenario we used the \CRANpkg{RDML} package to read-in data in 
standardized format. Further processing of amplification curve data was 
performed with the \CRANpkg{chipPCR} package and amplification curve data were 
analysed with the \CRANpkg{MBmca} package. Cq, quantification cycle; $T_{M}$, 
melting temperature. The \CRANpkg{dpcR} package can be embedded in the analysis 
of digital PCR experiments.
} \label{figure:workflow}
\end{figure}

\section{Setting-up a working environment}

We recommend to perform the scripting in a dedicated integrated development 
environment (IDE) and graphical user interface (GUI) such as \pkg{RKWard} 
\citep{rodiger_rkward_2012}, 
\pkg{Rstudio}\footnote{\url{http://www.rstudio.com/}} or related technologies 
\citep{Valero_2012}. Benefits of IDE's with GUI include syntax-highlighting, 
auto completion and function references for rapid prototyping of workflows.

Typically the qPCR analysis will start with data from a commercial platform. 
Most platforms have an option to export a CSV file or spreadsheets application 
file (e.g., *.xls, *.odt). The details for the import data has been described 
elsewhere \citep{RDCT2010c, rodiger_rkward_2012}. To keep the example sections 
compact we have chosen to load datasets from the \CRANpkg{qcpR} package 
\citep{ritz_2008, spiess_2008} (v.~1.4.0) and \CRANpkg{RDML} package to our 
workspace. In this study we used the \CRANpkg{RDML} package (v.~0.4-2) for data 
read-in. The data were measured with a CFX96 System (Bio-Rad) and then exported 
as RDML v1.1 format file. The \CRANpkg{chipPCR} package (v.~0.0.8-3) was used 
for data preprocessing, quality control and  the calculation of the 
quantification cycle (Cq). The Cq is a quantitative measure, which represents 
the number of cycles needed to reach a user defined threshold fluorescence 
signal level. Typically Cq are determined identically the exponential phase of a 
qPCR reaction. Several Cq methods have been described \citep{ruijter_2013}. In 
this study we have chosen the second derivative maximum method ($Cq_{SDM}$). Due to 
the ubiquitous use we used in a example also the ``Cycle threshold'' ($Cq_{Ct}$) method.

In a perfect qPCR reaction, the amount of amplicon doubles ($2^{n}$; n = cycle 
number) at each cycle. Here the amplification efficiency (AE) is 100~\%. 
However, in reality, numerous factors cause an inhibition of the amplification 
(AE < 100~\%).  The AE can determined be the relation of the Cq value depending 
on the sample input quantity as detailed described in \citep{roediger_chippcr_2014}.

In \citet{roediger_RJ_2013} we described the application of R for the analysis of 
melting curve experiments on the surface of microbeads. Since the mathematical 
foundation for melting curve analysis (MCA) is identical between all platforms 
we applied the functions from the \CRANpkg{MBmca} package 
\citep{roediger_RJ_2013} for an analysis of the target specific melting 
temperature ($T_{M}$) in our qPCR experiment. We used the \CRANpkg{MBmca} 
package (0.0.3-4) for analysis of melting curve data.

We complete our study with a simple example for the analysis of dPCR experiment. 
In particular, we used the \CRANpkg{dpcR} (0.1.3.1) to estimate the number of 
molecules in a sample.

\section{Results}

In this section we will try to show that R is a unified open software which fits 
the needs for (I) data analysis and presentation in research, (II) as software 
frame-work for novel technical developments, (III) as platform for teaching this 
new technology and (IV) serves as reference for statistical methods.

\subsection{Example one - qPCR and Amplification Efficiency Calculation}

The goal of our fist example was to calculate the Cq values and the AE from a 
qPCR experiment. Therefore, we used the ``guescini1'' dataset\footnote{Details 
of the experiment are described in \citet{guescini_2008}.} from the 
\CRANpkg{qcpR} package.

\begin{example}
# Collect information about the R session used for the analysis of the qPCR
# experiment.
current.session <- sessionInfo()

# Next we load the 'guescini1' dataset from the qpcR package the to
# workspace and assign it to the object tmp.
require(qpcR)
tmp <- guescini1

# Define the threshold value for the th.cyc function
Ct <- 0.05

# Define the diltuion of the sample DNA quantity for
# the calibration curve.

dil <- sapply((2:-4), function(i) {10^i})

# Preporcess the amplification curve data with the CPP function from the chipPCR
# package.
res.CPP <- cbind(tmp[, 1], apply(tmp[, -1], 2, function(x) {
    CPP(tmp[, 1], x, trans = TRUE, method.norm = "minm", bg.range = c(1,7))\$y.norm
}))

Cq.Ct <- apply(tmp[, -1], 2, function(x) {th.cyc(res.CPP[, 1], x, r = Ct)[1]})
Cq.SDM <- apply(tmp[, -1], 2, function(x) {summary(inder(res.CPP[, 1], x))[2]})

pdf("dilution_Cq.pdf", width = 9.5, height = 12)
layout(matrix(c(1,2,3,3,4,5), 3, 2, byrow = TRUE))

matplot(tmp[, -1], type = "l", lty = 1, col = 1, xlab = "Cycle", 
	    ylab = "RFU", main = "Raw data")
legend("topleft", "A", cex = 3, bty = "n")

matplot(res.CPP[, -1], type = "l", lty = 1, col = 1, xlab = "Cycle", 
	ylab = "RFU", main = "Pre-processed data")
legend("topleft", "B", cex = 3, bty = "n")
abline(h = Ct, col = "red", lwd = 2)

plot(Cq.SDM, Cq.Ct, xlab = "Ct method", ylab = "SDM method", 
     main = "Comparison of Cq methods")
abline(res.Cq)
legend("topleft", "C", cex = 3, bty = "n")

plot(effcalc(dil, t(matrix(Cq.Ct, nrow = 12, ncol = 7))), CI = TRUE)
legend("topright", "D", cex = 3, bty = "n")

plot(effcalc(dil, t(matrix(Cq.SDM, nrow = 12, ncol = 7))), CI = TRUE)
legend("topright", "E", cex = 3, bty = "n")
\end{example}

\begin{figure}[htbp]
  \centering
%   \includegraphics[clip=true,trim=0.1cm 0.6cm 1cm 1.8cm, width=16cm]{figures/plotCurves.pdf}
  \includegraphics[clip=true, width=14cm]{figures/dilution_Cq.pdf}
  \caption{Analysis of the amplification curve data of the calibration curve 
samples. \strong{(A)} Visual inspection of the raw data from the ``guescini1'' 
dataset. The qPCR curves display a broad variation in plateau fluorescence (38 
-- 62 RFU). The red horizontal line indicates the fluorescence level (0.05) used 
for the calculation of the Cq by the ``cycle threshold'' method. \strong{(B)} 
the \code{CPP} function from the \CRANpkg{chipPCR} was sued to baseline the 
data, to smooth the data with Savitzky-Golay smoothing filter and to normalize 
the data between 0 and 1. \strong{(C)} The Cq values were calculated for the 
second derivative maximum (SDM) method (\code{inder}, \CRANpkg{chipPCR}) and the 
cycle threshold method (Ct) (\code{th.cyc}, \CRANpkg{chipPCR}). The threshold 
value was set to $r~=~0.05$. The Cq values from the SDM and Ct method were 
plotted and analysed by a linear regression $R^{2}~=~0.9945$ ($P < 2.2^{-16}$) 
and Pearson's $r~=~0.9972605$ ($P < 2.2^{-16}$). The Cq values from \strong{(D)} 
the Ct method and \strong{(E)} the SDM method were automatically analysed with 
the \code{effcalc} (\CRANpkg{chipPCR}) function.}
  \label{figure:dilution_Cq}
\end{figure}

\subsection{Example two - qPCR and Melting Curve Analysis}
In this study we used the \CRANpkg{RDML} package to read the qPCR experiment. A 
good practice for reproducible research is to track the package versions and 
environment used during the analysis. The function $sessionInfo()$ from the 
\CRANpkg{utils} package provides this information. Assuming that the analysis 
starts with a clean R session it is possible to assign the required packages to 
an object only, as shown in our example below\footnote{The reproducibility of 
research can be further improved by using dedicated tools. For example, 
\CRANpkg{archivist} package allows not only stores and recovers crucial data, 
but also preserves metadata of saved objects.}.

\begin{example}
# Load the required packages for the data import and analysis.
# Import the qPCR and melting curve data via the RDML package
require(RDML)

# Load the chipPCR package for the pre-processing and curve data quality
# analysis.
require(chipPCR)

# Load the MBmca package for the melting curve analysis.
require(MBmca)

# Collect information about the R session used for the analysis of the
# qPCR experiment.

current.session <- sessionInfo()

# Load lc96_bACTXY.rdml dataset form RDML package and assign the data to the 
# object LC96.dat. The data were measured with CFX96 (Bio-Rad). The 
# data set contains qPCR data with four targets and two types.

PATH <- path.package("RDML")
filename <- paste(PATH, "/extdata/", "lc96_bACTXY.rdml", sep ="")
lc96 <- RDML(filename)
\end{example}

Next we inspected and pre-processed a subset of the amplification curve data 
entirely with functionality from the \CRANpkg{chipPCR} package. The 
\code{plotCurves} function was used to get an overview of the curvatures. The 
data indicated a baseline shift in all curves with a slight negative trend 
(Figure~\ref{figure:plotCurves}). Therefore, we used the 

\begin{example}
# Fetch cycle dependent fluorescence for HEX channel of beta actin (bACT).
tmp.std <- lc96$qPCR[["FAM@bACT"]]$std


# Use plotCurves function from the chipPCR package to get an overview of the 
# amplification curve samples.

plotCurves(tmp.std[, 1], tmp.std[, -1], type = "l")

\end{example}


\begin{figure}[htbp]
  \centering
  \includegraphics{figures/plotCurves.pdf}
  \caption{Analysis of the amplification curve data of the calibration curve 
samples by the \code{plotCurves} function from the \CRANpkg{chipPCR} package.}
  \label{figure:plotCurves}
\end{figure}

\begin{figure}[htbp]
  \centering
  \includegraphics[clip=true, width=16cm]{figures/amp_melt.pdf}
  \caption{by the \code{diffQ} function from the \CRANpkg{MBmca} package.}
  \label{figure:amp_melt}
\end{figure}

\subsection{Example three - Isothermal Amplification}

A quantitative isothermal amplifcation by Helicase Dependent Amplification (HDA) 
was performed for the of pCNG1 using the VideoScan platform 
\citep{rodiger_highly_2013}. The dataset was taken form the \CRANpkg{chipPCR} 
package. The HDA was performed at 65~\textcelsius. Two concentrations of input 
DNA were used. Number of variables: 351 Number of measurements: 5 Data set: C85:


\begin{figure}[htbp]
  \centering
  \includegraphics[clip=true, width=16cm]{figures/qIA.pdf}
  \caption{by the \code{th.cyc} function from the \CRANpkg{chipPCR} package.}
  \label{figure:qIA}
\end{figure}


\subsection{Example four - digital PCR}

We have developed the \CRANpkg{dpcR} package for analysis and presentation of 
digital PCR experiments. The \CRANpkg{dpcR} package can be used to build 
custom-made analysers and provides structures to be openly extended by the 
scientific community. Simulations and predictions of binomial and Poisson 
distributions, commonly used theoretical models of dPCR, statistical data 
analysis methods, plotting facilities and report generation tools are part of 
the package \citep{pabinger_2014}. Here, we show briefly an example for the 
\CRANpkg{dpcR}\footnote{Selected functionality was implemented as 
interactive\CRANpkg{shiny} GUI application to make the software accessible for 
users who are not fluent in R and but also for experts who which to automatize 
routine tasks. Details and examples of the \CRANpkg{shiny} web application 
framework for R can be found at \url{http://shiny.rstudio.com/}. We implemented 
flexible yet simple user interfaces, which run the analyses and graphical 
representation into interactive web applications either as service on a web 
severer or on a local machine without knowledge of HTML or ECMAScript (see 
\CRANpkg{dpcR} manual). The interface is designed in a cascade workflow approach 
(Data import $\rightarrow$ Analysis $\rightarrow$ Output $\rightarrow$ Export) 
with interactive users choice on input data, methods and parameters using 
typical GUI elements such as sliders, drop-downs and text fields. An example can 
be found at \url{https://michbur.shinyapps.io/dpcr_density/}. This approach 
enables the automatized outputs of R objects in combined plots, tables and 
summaries.}.

\begin{figure}[htbp]
  \centering
%   \includegraphics[clip=true,trim=0.1cm 0.6cm 1cm 1.8cm, width=16cm]{figures/plotCurves.pdf}
  \includegraphics[clip=true, width=14cm]{figures/dpcR.pdf}
  \caption{\code{dpcr\_density} function from the \CRANpkg{dpcR} package.}
  \label{figure:dpcR}
\end{figure}

\section{Discussion and Conclusion}

This study gave a brief introduction how to perform a qPCR, qIA or dPCR analysis 
with R based on packages available from CRAN. In addition we briefly referenced 
to a vast collection of additional packages available from CRAN and 
Bioconductor. The packages may be considered building blocks (libraries) to 
create what users want and need. We showed that an automatic research with R 
offers powerful means for statistical analysis and visualization. The software 
is not tied to a vendor or specific application (e.g, chamber or droplet based 
digital PCR, capillary or plate qPCR). It should be quite easy even for an 
inexperienced user to define a workflow and to setup environment for specific 
needs in a broad range of technical settings (Figure~\ref{figure:options}). R 
enforces no monolithic integration. We claim that the modular structure of R 
packages allows user to perform flexible data analysis adjusted to their needs 
and to design frameworks for high-throughput analysis. R allows to access and 
reuse code for the creation of reports in various formats (e.g., HTML, PDF). 
Most of the software is cross-platform open source software and is freely 
available from CRAN or Bioconductor. Despite the fact that R is free of charge 
it is quite possible to build commercial applications. The packages cover 
implementation of novel approaches and peer-reviewed analysis methods. R 
packages are an open environment to adopt to the growing knowledge in dPCR and 
qPCR. Therefore, we argue that R may provide a structure for standardized 
nomenclature and serve as reference in qPCR and dPCR analysis. Speaking about 
openness, it is important to emphasize that main advantage is the software is 
transparent at any time for anybody. Thus, it is possible to track numerical 
errors. 

\url{http://michbur.github.io/pcRuniveRsum/}

A serve disadvantage of R is the lack of comprehensive GUIs for qPCR analysis. 
Other and we believe that a graphical user interface (GUI) is a key technology 
to spread the use of R in bioanalytical sciences. The command-line structure 
makes R ``inaccessible'' for many novices. This we support the attempt that 
automatic routines are made accessible via GUIs \citep{rodiger_rkward_2012}. 
However, work in this has has recently started and is still under development.

\begin{figure}[htbp]
  \centering
  \includegraphics[clip=true, width=10cm]{figures/options.png}
  \caption{Deployment of R applications for the qPCR and dPCR experiments. 
\strong{(A)} R is typically run from a desktop computer an operated by an 
GUI/IDE application such a \pkg{Rstudio} or \pkg{RKWard}. This approach is 
provides a flexible workflow for individuals. \strong{(B)} Another approach is 
to run R with specific applications on a local server. Such scenarios are useful 
for the deployment within research departments or cooperate units. \strong{(C)} 
Cloud computing (CC) provides shared and scalable computing capacity (e.g., 
computing capacity, application software) and storage capacity (e.g., databases) 
as a service to an individual user or a community Service categories include: 
Infrastructure-as-a-Service (IaaS), Platform-as-a-Service (PaaS) and 
Software-as-a-Service (SaaS) over a network. Providers of CC manage the 
infrastructure and resources to achieve coherence and economies of scale similar 
to a utility over a network (typically the Internet).}
  \label{figure:options}
\end{figure} 

\section{Acknowledgment}

Part of this work was funded by the BMBF InnoProfile-Transfer-Projekt 03 IPT 
611X. We would like to thank R community. Part of this work was funded by the Russian Ministry of 
Education and Science (project No. RFMEFI62114X0003) and with usage of scientific equipment of 
Center for collective use ``Biotechnology'' at All-Russia Research Institute of Agricultural Biotechnology.

\bibliography{Roediger_2014_R}

\address{Stefan R\"odiger (corresponding author)\\
  Faculty of Natural Sciences\\
  Brandenburg University of Technology Cottbus--Senftenberg\\
  Senftenberg\\
  Germany}
\email{Stefan.Roediger@hs-lausitz.de}

\address{Micha\l{} Burdukiewicz\\
  University of Wroclaw\\
  Department of Biotechnoloy\\
  Wroclaw\\
  Poland}
\email{michalburdukiewicz@gmail.com}

\address{Konstantin Blagodatskikh\\
  All-Russia Research Institute of Agricultural Biotechnology\\
  Center for collective use ``Biotechnology''\\
  Moscow\\
  Russia}
\email{k.blag@yandex.ru}

\address{Peter Schierack\\
  Faculty of Natural Sciences\\
  Brandenburg University of Technology Cottbus--Senftenberg\\
  Senftenberg\\
  Germany}
\email{Peter.Schierack@hs-lausitz.de}